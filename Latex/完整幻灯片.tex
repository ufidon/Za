\documentclass[UTF8]{ctexbeamer}
\xeCJKsetup{CJKmath=true}
\setCJKmainfont{NotoSerifCJKsc-SemiBold} % CJK font
\usetheme{Madrid}
\usecolortheme{lily}
\usepackage{amsmath,amsfonts,amssymb, amsthm}
\usepackage[linesnumbered,ruled]{algorithm2e}
\usepackage[absolute,overlay]{textpos}
\usepackage{graphicx, color}
\usepackage{tikz}
\usepackage[utf8]{inputenc}
\usepackage{multicol}
\usepackage{default}
\usepackage{pgfplots}
\usepackage{pgfplotstable}
\usepackage{verbatim}
\usepackage{booktabs}
\usepackage{pdfpages}
\usepackage{hyperref}
\usepackage[export]{adjustbox}

\usepackage{etoolbox}

% Defining a new coordinate system for the page:
%
% --------------------------
% |(-1,1)    (0,1)    (1,1)|
% |                        |
% |(-1,0)    (0,0)    (1,0)|
% |                        |
% |(-1,-1)   (0,-1)  (1,-1)|
% --------------------------
\makeatletter
\def\parsecomma#1,#2\endparsecomma{\def\page@x{#1}\def\page@y{#2}}
\tikzdeclarecoordinatesystem{page}{
	\parsecomma#1\endparsecomma
	\pgfpointanchor{current page}{north east}
	% Save the upper right corner
	\pgf@xc=\pgf@x%
	\pgf@yc=\pgf@y%
	% save the lower left corner
	\pgfpointanchor{current page}{south west}
	\pgf@xb=\pgf@x%
	\pgf@yb=\pgf@y%
	% Transform to the correct placement
	\pgfmathparse{(\pgf@xc-\pgf@xb)/2.*\page@x+(\pgf@xc+\pgf@xb)/2.}
	\expandafter\pgf@x\expandafter=\pgfmathresult pt
	\pgfmathparse{(\pgf@yc-\pgf@yb)/2.*\page@y+(\pgf@yc+\pgf@yb)/2.}
	\expandafter\pgf@y\expandafter=\pgfmathresult pt
}
\makeatother

\makeatletter
\apptocmd{\beamer@@frametitle}{\write\@auxout{\string\@writefile{frm}{\string\frametitleentry{\the\c@framenumber}{#1}{#2}}}}{}{}
\newcommand*{\frametitleentry}[3]{\@namedef{frametitleshort#1}{#2}\@namedef{frametitle#1}{#3}}
\AtEndDocument{\if@filesw\newwrite\tf@frm\immediate\openout\tf@frm\jobname.frm\relax\fi}
\@input{\jobname.frm}
\newcommand*{\insertpreviousframetitle}[1][1]{\bgroup\advance\c@framenumber by -#1\relax\@ifstar{\@nameuse{frametitleshort\the\c@framenumber}\egroup}{\@nameuse{frametitle\the\c@framenumber}\egroup}}
\newcommand*{\insertnextframetitle}[1][1]{\bgroup\advance\c@framenumber by #1\relax\@ifstar{\@nameuse{frametitleshort\the\c@framenumber}\egroup}{\@nameuse{frametitle\the\c@framenumber}\egroup}}

\newcommand{\xexample}[1]{\textcolor[rgb]{0.5, 0, 1}{#1}}
\newcommand{\xalert}[1]{\textcolor[rgb]{1 0.5 0}{#1}}
\newcommand{\xterm}[1]{\textcolor{blue}{#1}}
\newcommand{\xquestion}[1]{\textcolor[rgb]{1, 0.25002, 0}{#1}}
\newcommand{\xsolution}[1]{\textcolor[rgb]{0.50018, 0, 1}{#1}}
\makeatother



\graphicspath{{Figs/}{figs/}}

\title[Pore Formation Energy]{Comparative Study of Pore Formation Energy by High Intensity, Nanosecond Electrical Pulse}
\author[X. Wang]{H. Qiu \inst{1} \and X. Wang \inst{2} \and A. Choi \inst{3} \and W. Zhao\inst{4} }
\institute[SEMO]{\inst{1}Dept of Engineering Technology, Fort Valley State University \and %
	\inst{2}Dept of Computer Sciences, Southeast Missouri State University \and %
	\inst{3}Dept of Electrical \& Computer Engineering, Mercer University \and %
				     \inst{4}Dept of Electrical Engineering \& Computer Science }
\date[EMBC 2018]{40th International Conference of the IEEE Engineering in Medicine and Biology Society}

\begin{document}

\begin{frame}
\maketitle
%\begin{textblock*}{.32\textwidth}(0.7cm,1.5cm) % {block width} (coords)
%\includegraphics[width=0.6\textwidth]{elogo.jpg}
%\end{textblock*}
%\begin{textblock*}{.3\textwidth}(0.80\textwidth,1.5cm) % {block width} (coords)
%\includegraphics[width=0.6\textwidth]{ilogo.jpg}
%\end{textblock*}
\end{frame}

%\addtobeamertemplate{frametitle}{}{%
%%\begin{textblock*}{100mm}(0.96\textwidth,0cm)
%%\includegraphics[height=1cm,width=1.38cm]{logo}
%%\end{textblock*}}
%\begin{tikzpicture}[remember picture,overlay, every node/.style={anchor=north east}]
%\node at (page cs:1,1) {\includegraphics[height=1cm,width=1.38cm]{logo}};
%\end{tikzpicture}
%}

\addtobeamertemplate{background canvas}{}{
\begin{tikzpicture}[remember picture,overlay, every node/.style={anchor=north east}]
\node[opacity=0.3] at (page cs:1,1) {\includegraphics[height=1cm,width=1.38cm]{logo}};
\end{tikzpicture}
}


\setbeamertemplate{footline}
{
  \leavevmode%
  \hbox{%
  \begin{beamercolorbox}[wd=.4\paperwidth,ht=2.25ex,dp=1ex,center]{author in head/foot}%
    \usebeamerfont{author in head/foot}\insertsection
  \end{beamercolorbox}%
  \begin{beamercolorbox}[wd=.6\paperwidth,ht=2.25ex,dp=1ex,right]{date in head/foot}%
    \usebeamerfont{date in head/foot}\insertnextframetitle\hspace*{2em}
    \insertframenumber{} / \inserttotalframenumber\hspace*{2ex} 
  \end{beamercolorbox}}%
  \vskip0pt%
}

\begin{frame}{Outline}
\tableofcontents
\end{frame}

\section{Background}
\frame{\tableofcontents[currentsection]}

\begin{frame}{What is Electroporation?}
	Electroporation is a biophysical process which enhances permeability of cell membranes by exerting an external electric field. It
	\begin{itemize}
		\item uses high-intensity, short-duration electrical pulses
		\item creates pores in the lipid bilayer leading to rapid increase of electrical conductivity
		\item promotes electrically-triggered intra-cellular calcium release
		\item penetrates into mitochondria, damages DNA and micro-organisms, etc.
	\end{itemize}
\end{frame}

\begin{frame}{Electroporation Illustration}
\includegraphics[width=\textwidth]{Electroporation.jpg}
\footnote{Image from https://www.creative-biolabs.com/car-t/electroporation.htm}
\end{frame}

\begin{frame}{Application of Electroporation}
Electroporation can be applied to
\begin{itemize}
	\item 除掉坏死心肌
	\item $ \int_{下限}^{上限} 力(r)dr $
	\item 助愈
\end{itemize}
\begin{tikzpicture}[remember picture,overlay, every node/.style={anchor=north east}]
\node[opacity=0.6] at (page cs:0,0) {你好$ \int_{下限}^{上限} 力(r)dr $};
\end{tikzpicture}
%\begin{tikzpicture}[remember picture,overlay, every node/.style={anchor=north east}]
%\node at (page cs:0.8,-0.1) {\includegraphics[height=0.4\textheight]{f1.png}};
%\end{tikzpicture}
\end{frame}


\section{Build the Model}
\frame{\tableofcontents[currentsection]}
\begin{frame}{Model Evolution}
\begin{itemize}
	\item Early continuum models utilize the Smoluchowski theory for time-dependent evolutions using the pore formation energy
	\item Later improvements to the continuum model include
	\begin{itemize}
		\item the role of variable surface tension
		\item Born energy corrections
		\item current flow through the generated pores
		\item electric field contributions to the pore formation energy
		\item Maxwell stress
	\end{itemize}
\end{itemize}

\end{frame}

\begin{frame}{Pore Shape in the Model}
In recent research, it shows that pore shapes, cylindrical and toroidal, also affect electroporation.
In this paper, using numerical method, we
\begin{itemize}
	\item compare cylindrical and toroidal pores based on continuum theory
	\item explore the role of membrane curvature in electroporation
\end{itemize} 
\end{frame}

\begin{frame}{Build the Model}
\begin{itemize}
	\item The Maxwell stress, on the cell membrane altering the pore-radius dependent energy, can be calculated based 	on the surrounding electric field distribution
	\item The total pore formation energy $ E_{pore}(r) $ can be expressed as: 
	
	\scalebox{0.9}{%
	$ 
	E_{pore}(r) = \begin{cases}
	2\pi \gamma r - \pi r^2 \Gamma - \frac{\pi}{2h}(\epsilon_w - \epsilon_m)V_m^2 r^2 + E_{strain}(r) &  r>r_{hh}\\
	2\pi hr\Gamma_h \frac{I_1(\frac{r}{\lambda})}{I_0(\frac{r}{\lambda})}- \pi r^2 \Gamma - \frac{\pi}{2h}(\epsilon_w - \epsilon_m)V_m^2 r^2 + E_{strain}(r) & r\le r_{hh}
	\end{cases}
	$
	}

	\item  The membrane curvature plays a role in the strain energy:
	\[ 
	E_{strain}(r) = -\int_{0}^{r} F(r)\cos(\alpha)dr
	\]
\end{itemize}
\end{frame}

\begin{frame}{The Continuum Model and its Parameters}
\begin{tikzpicture}[remember picture,overlay, every node/.style={anchor=center}]
\node at (page cs:-0.6,0) {\includegraphics[height=0.45\textheight]{f1.png}};
\end{tikzpicture}
\begin{tikzpicture}[remember picture,overlay, every node/.style={anchor=center}]
\node at (page cs:0.4,0) {\includegraphics[height=0.4\textheight]{t1.png}};
\end{tikzpicture}
%\includegraphics[width=0.45\textwidth]{f1.png}
%\includegraphics[width=0.45\textwidth]{t1.png}
\end{frame}



\section{Simulation Results and Discussion}
\frame{\tableofcontents[currentsection]}
\begin{frame}{Surface Electric Potential Distribution of a Toroidal Pore}
The transmembrane potential is always kept fixed as a constant value:
	\begin{itemize}
		\item 1 volt for critical electroporation
		\item 0.3 volt for early stage of complete pore formation
	\end{itemize}
	
	\vspace{0.5cm}
	\begin{columns}
		\begin{column}{0.45\textwidth}
			\includegraphics[height=0.45\textheight]{f2.png}
		\end{column}
	\begin{column}{0.45\textwidth}
		The electric potential on the pore surface is distributed uniformly. The voltage difference
		of the top surface and the bottom surface is on average around 1 volt, which is often considered as a critical value of pore formation.
	\end{column}
	\end{columns}
\end{frame}

\begin{frame}{The Stress Force on the Surface of Pores}
The force on the surface of pores is increasing as the pore size gets larger.
\vspace{0.5cm}
	\begin{columns}
		\begin{column}{0.4\textwidth}
			\includegraphics[height=0.45\textheight]{f3.png}
		\end{column}
	\hspace{-1.5cm}
		\begin{column}{0.6\textwidth}
			\begin{itemize}
				\item The force is within the limit of 2 nN for a toroidal pore and 1 nN for a cylindrical pore
				\item A toroidal pore produces stronger force than a cylindrical pore at varied pore sizes
				\item The increment of the force of a toroidal pore is slightly faster than a cylindrical pore
			\end{itemize}
		\end{column}	
	\end{columns}
\end{frame}

\begin{frame}{ $ E_{strain}  $ of pores}
The magnitude of formation energy of pores is increasing as the pore size gets larger.
\vspace{0.5cm}
\begin{columns}
	\begin{column}{0.4\textwidth}
		\includegraphics[height=0.45\textheight]{f4.png}
	\end{column}
	\hspace{-1.5cm}
	\begin{column}{0.6\textwidth}
		\begin{itemize}
			\item The pore energy is within the limit of $ 400 kT $ (unit $ kT , k$: Boltzmann constant, $ T=300K $) for a toroidal pore and the limit of $ 330 kT $ for a cylindrical pore
			\item A toroidal pore has more energy than a cylindrical pore at all different pore sizes
			\item The energy of a toroidal pore increases more rapidly than a cylindrical pore
		\end{itemize}
	\end{column}	
\end{columns}
\end{frame}

\begin{frame}{Force \& Energy of the Pores}
\centering
\includegraphics[width=0.8\textwidth]{t2.png}
\end{frame}


\begin{frame}{}
	\vspace{1cm}
	\centering {\Huge \textbf{\color[rgb]{0.0, 0.72, 0.92}{Thank you for your time!}}}
	\begin{figure}[htbp]
	\centering
	\includegraphics[height=0.4\textheight]{question}
	\end{figure}
	%\centering {\Huge \textbf{\color[rgb]{0.0, 0.72, 0.92}{Thank the search committee for giving me this valuable chance!}}}
\end{frame}


\end{document}