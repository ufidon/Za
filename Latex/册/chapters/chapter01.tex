\chapter{The First Chapter}

\begin{tabular}{l}
    \hline
    “天地玄黄,宇宙洪荒。日月盈昃,辰宿列张。”\footnotemark \\
    \hline
\end{tabular}
\footnotetext{表格里的名句出自《千字文》。}
    
\marginpar{\footnotesize 边注较窄,不要写过多文字,最好设置较小的字号。}

《木兰诗》:
\begin{quotation}
	万里赴戎机,关山度若飞。
	朔气传金柝,寒光照铁衣。
	将军百战死,壮士十年归。
	
	归来见天子,天子坐明堂。
	策勋十二转,赏赐百千强。......
\end{quotation}

\begin{quote}
	万里赴戎机,关山度若飞。
	朔气传金柝,寒光照铁衣。
	将军百战死,壮士十年归。
	
	归来见天子,天子坐明堂。
	策勋十二转,赏赐百千强。......
\end{quote}

\begin{verse}
	万里赴戎机,关山度若飞。
	朔气传金柝,寒光照铁衣。
	将军百战死,壮士十年归。
	
	归来见天子,天子坐明堂。
	策勋十二转,赏赐百千强。......
\end{verse}

\begin{verbatim}
#include <iostream>
int main()
{
  std::cout << "你好,高那德!"
  << std::endl;
  return 0;
}
\end{verbatim}

\begin{verbatim*}
#include <iostream>
int main()
{
  std::cout << "你好,高那德!"
  << std::endl;
  return 0;
}
\end{verbatim*}

\verb|\LaTeX| \\
\verb+(a || b)+ \verb*+(a || b)+


\begin{lstlisting}
for i:=maxint to 0 do
begin
{ 做点啥 }
end;
Write(’Case insensitive ’);
WritE(’Pascal keywords.’);
\end{lstlisting}


基线
\begin{tabular}{|c|}
	表腰\\
	与基线对齐\\
\end{tabular}
\begin{tabular}[t]{|c|}
	表头\\
	与基线对齐\\
\end{tabular}
\begin{tabular}[b]{|c|}
	表脚\\
	与基线对齐\\
\end{tabular}
基线\par

下一段:万里赴戎机,关山度若飞。
朔气传金柝,寒光照铁衣。
将军百战死,壮士十年归。

\def\夏树{\Summertree}
\夏树

\begin{tabular}{|@{\Springtree}r@{} @{:}l r@{\Wintertree}|}
	\hline
	1 & 1 & 一 \\
	11 & 3 & 十一 \\
	\hline
\end{tabular}

\begin{tabular}
	{>{\centering\arraybackslash}p{9em}}
	\hline
	万里赴戎机,关山度若飞。
	朔气传金柝,寒光照铁衣。 \\
	\hline
\end{tabular}

\newcommand\木兰辞{万里赴戎机}

\begin{tabular}{cp{2em}m{2em}b{2em}}
	\hline
	基线 & \木兰辞 & \木兰辞 & \木兰辞 \\
	\hline
\end{tabular}

\begin{tabularx}{14em}%
	{|*{5}{>{\centering\arraybackslash}X|}}
	\hline
	甲 & 乙 & 丙 & 丁 & 戊 \\ \hline
	己 & 庚 & 辛 & 壬 & 癸 \\ \hline
\end{tabularx}

\begin{tabular}{|c|c|c|}
	\hline
	4 & 9 & 2 \\ 
	\cline{2-3}
	3 & 5 & 7 \\ 
	\cline{1-1}
	8 & 1 & 6 \\ 
	\hline
\end{tabular}

\begin{tabular}{cccc}
	\toprule
	& \multicolumn{3}{c}{数字} \\
	\cmidrule{2-4}
	& 1 & 2 & 3 \\
	\midrule
	Alphabet & A & B & C \\
	Roman
	& I & II& III \\
	\bottomrule
\end{tabular}

\begin{tabular}{cccc}
	\hline\\
	& \multicolumn{3}{c}{数字} \\
	\cmidrule{2-4}
	& 1 & 2 & 3 \\
	\hline\\
	字母 & A & B & C \\
	欧数
	& I & II& III \\
	\hline\\
\end{tabular}

\begin{tabular}{|*{3}{c|}}
	\hline
	甲 & 乙 & 丙 \\\hline
	\multicolumn{2}{|c|}{甲乙} & \multicolumn{1}{r|}{丁}\\\hline
	甲 & \multicolumn{2}{c|}{乙丙} \\\hline
\end{tabular}

\begin{tabular}{ccc}
	\hline
	\multirow{2}{*}{项} & \multicolumn{2}{c}{值}\\
	\cline{2-3}
	& 首 & 末 \\ \hline
	甲 & 乙 & 丙 \\\hline
\end{tabular}

|\mbox{ 庚、辛、壬、癸}|\\
|\makebox[10em]{ 庚、辛、壬、癸。}|\\
|\makebox[10em][l]{ 庚、辛、壬、癸。}|\\
|\makebox[10em][r]{ 庚、辛、壬、癸。}|\\
|\makebox[10em][s]{ 庚、辛、壬、癸。}|

\framebox[10em][r]{测试盒。}\\[1ex]
\setlength{\fboxrule}{1.6pt}
\setlength{\fboxsep}{1em}
\framebox[10em][r]{\phantom{测}}
\framebox[10em][r]{测试盒。}

三字经:\parbox[t]{3em}%
{人之初 性本善 性相近 习相远}
\quad
千字文:\lower0.4ex\hbox{
\begin{minipage}[b][2ex][t]{4em}
	天地玄黄 宇宙洪荒
\end{minipage}}

\meaning\XeteX

微页\fbox{\begin{minipage}{15em}%
		这是一个垂直盒子的测试。
		\footnote{脚注来自 微页。}
\end{minipage}}

黑 \rule{12pt}{4pt} 盒.
上些 \rule[4pt]{6pt}{8pt} 与
下些 \rule[-4pt]{6pt}{8pt} 盒.
一条 \rule[-.4pt]{3em}{.4pt} 线.
\hrule

\TPGrid[0mm,0mm]{10}{10}
\TPShowGrid{10}{10}

\begin{textblock}{1}(0,0)
\textblocklabel{第二块}
任意位置放文本。
\end{textblock}

\begin{tikzpicture}[remember picture,overlay,every node/.style={anchor=center}]
\node at (page cs:0.5,0.3) {0.5,0.3};
\node at (page cs:-0.25,0.3) {-0.25,0.3};
\node at (page cs:0,0) {0,0};
\draw(page cs:-0.25,0) -- (page cs:.75,-0.5);
\draw[thick] (page cs:-1,-1) rectangle (page cs:1,1);
\end{tikzpicture}

\lettrine[lines=3,lhang=0.2,loversize=0.25]{关}{山}度若飞

